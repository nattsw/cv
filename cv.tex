%!TEX TS-program = xelatex
\documentclass[]{friggeri-cv}

\usepackage{marvosym} % Allows the use of symbols
\usepackage{amsmath} % Allows the use of symbols
\usepackage{enumitem}
%\addbibresource{bibliography.bib}

\begin{document}
\header{shawn}{tan}
{aspiring computer scientist}
% In the aside, each new line forces a line break
\begin{aside}
	\section{online}
	\Email~ \href{mailto:shawn@wtf.sg}{shawn@wtf.sg}
	\href{http://blog.wtf.sg}{http://blog.wtf.sg}
	\href{http://facebook.com/shawntan.171287}{fb://shawntan.171287}
	\href{http://github.com/shawntan}{shawntan@github}
	\href{http://sg.linkedin.com/in/tanshawn}{tanshawn@linkedin}
	\section{languages}
	english (proficient)
	mandarin
	\section{programming}
	{\color{red} $\varheartsuit$} JavaScript, Java
	Python, Ruby, C, Perl
	\section{technical skills}
	HTML \& CSS, \LaTeX,
	Linux System Administration
\end{aside}
\section{about}
\begin{tabular}{ p{0.45cm} p{6.5cm} p{0.45cm} p{6.5cm} }
	\Male 	& Tan Jing Shan, Shawn	 & \Sagittarius &  DOB: 17/12/1987\\
	\\
	\Letter & 1082 Washington Street, Mountain View & \Mobilefone  & (+1) 
	650-966-4596  {\footnotesize\textit{(current)}}\\
			& California 94043 & & \\
	\Letter & 5 Jalan Asas, Singapore 678763 & \Mobilefone & (+65) 93877026 \\
\end{tabular}

\section{experience}

\begin{entrylist}
	\entry
	{2013 - now}
	{Semantics3}
	{Software Engineer}
	{}
	\entry
	{2008 - 2009}
	{Singapore Armed Forces}
	{Manpower Officer}
	{Administrative position managing National Servicemen}
	\entry
	{04 - 06 2007}
	{\href{http://www.np.edu.sg/ict/facilities/rhymes/Pages/loc\_rhymes.aspx}{RHyMeS Center, Ngee Ann Polytechnic}}
	{Research Assistant}
	{Development of a conference management system using active RFID tags for 
	tracking.}
	\entry
	{01 - 03 2007}
	{Ngee Ann Polytechnic}
	{Teaching Assistant}
	{
		\begin{itemize}[itemsep=0pt,topsep=0pt]
			\item Provided technical support for students working on the RHyMeS 
				project.
			\item Facilitated workshops on how to use the RHyMeS SDK and API.
			\item  Taught students working on their projects how to use the Java 
				Swing UI API
		\end{itemize}
	}
\end{entrylist}

\section{education}

\begin{entrylist}
	\entry
	{2009 - 2012}
	{Bachelors of Computing (BComp)}
	{National University of Singapore}
	{
		Major in Computer Science - Upper 2$^{\text{nd}}$ Class Honours \\
		Special Programme in Computing (Turing Programme) \\
		Focus Area: Artificial Intelligence
	}
	\entry
	{2004 - 2007}
	{Diploma in Information Technology}{Ngee Ann Polytechnic, Singapore}
	{Specialisation in Software Engineering - Diploma with Merit}
	\entry
	{2003}
	{GCE 'O' Levels}{Mayflower Secondary School, Singapore}
	{}
\end{entrylist}


\section{organisations}

\begin{entrylist}
	\entry
	{2010 - 2012}
	{NUS Hackers}
	{Coreteam Member}
	{Duties include facilitating NUS Hacker activities: Hack \& Roll 2012, 
weekly Friday Hacks, and maintaining the download.nus.edu.sg mirror service}
	\entry
	{2003 - 2012}
	{Singapore Scouts Association}{Assistant Scout Leader}
	{Attached to the Mayflower Secondary School Boy Scouts Group. Duties include 
	facilitating camps, weekly meetings, and mentoring both scouts and 
ventures.}
\end{entrylist}
%\printbibsection{article}{article in peer-reviewed journal}
%\begin{refsection}
%	\nocite{*}
%	\printbibliography[sorting=chronological, type=inproceedings, title={international peer-reviewed conferences/proceedings}, notkeyword={france}, heading=subbibliography]
%\end{refsection}
%\begin{refsection}
%	\nocite{*}
%	\printbibliography[sorting=chronological, type=inproceedings, title={local peer-reviewed conferences/proceedings}, keyword={france}, heading=subbibliography]
%\end{refsection}
%\printbibsection{misc}{other publications}
%\printbibsection{report}{research reports}
\section{interests}
artificial general intelligence, machine learning, reinforcement learning,
graphical models,\\ natural language processing, data mining
\end{document}
