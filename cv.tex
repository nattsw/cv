%!TEX TS-program = xelatex
\documentclass[]{friggeri-cv}

\usepackage{marvosym} % Allows the use of symbols
\usepackage{amsmath} % Allows the use of symbols
\usepackage{enumitem}
%\addbibresource{bibliography.bib}

\begin{document}
\header{natalie}{tay}
{aspiring computer engineer}
% In the aside, each new line forces a line break
\begin{aside}
	\section{online}
	\Email~ \href{mailto:nat@lie.sg}{nat@lie.sg}
	%\href{http://blog.wtf.sg}{http://blog.wtf.sg}
	\href{https://www.facebook.com/natalie.tsw}{fb://natalie.tsw}
	\href{http://github.com/nattsw}{nattsw@github}
	\href{http://sg.linkedin.com/in/natalietsw}{natalietsw@linkedin}
	\section{languages}
	english (proficient)
	mandarin
	\section{programming}
	{\color{red} $\varheartsuit$} C\#
	C
	C++
	Objective-C
	Java (Android)
	VHDL
	JavaScript
	Assembly 	
	\section{technical skills}
	HTML \& CSS
	\LaTeX
	Git
	AutoCAD	
	Database Administration
\end{aside}
\section{about}
\begin{tabular}{ p{0.45cm} p{6.5cm} p{0.45cm} p{6.5cm} }
	\Female 	& Tay Shu Wen, Natalie	 & \Scorpio &  DOB: 16/11/1989\\
	\Letter & 434 Clementi Ave 3, Singapore 120434 & \Mobilefone & (+65) 86123211 \\
\end{tabular}

\section{experience}

\begin{entrylist}
	\entry
	{12 2012 - now   \ \ }
	{ST Electronics (Info-Comm)}
	{Software Engineer}
	{Defence Electronics Division
	\begin{itemize}[itemsep=0pt,topsep=0pt]
	\item Built a web server in C on an Atmel AVR UC3 microcontroller
	\item Handled high speed graphics rendering
	\item Project requirements analysis and solution sourcing
	\item Built a high level site scraper
	\end{itemize}}
	\entry
	{12 2011 }
	{ST Electronics (Info-Comm)}
	{Assoc Engineer}
	{Defence Electronics Division \\ Continued work on R\&D project with Defence Electronics Division}{}
	\entry
	{06 - 08 2011 }
	%{\href{http://www.np.edu.sg/ict/facilities/rhymes/Pages/loc\_rhymes.aspx}{RHyMeS Center, Ngee Ann Polytechnic}}
	{ST Electronics (Info-Comm)}
	{Assoc Engineer}
	{Defence Electronics Division \\Worked on R\&D project which required:
	\begin{itemize}[itemsep=0pt,topsep=0pt]
	\item Extensive knowledge on Human-Computer Interaction
	\item Optimisation of computer memory use
	\item Radio communications protocols
	\item Presentation skills
	\end{itemize}}
	\entry
	{04 - 08 2009}
	{ST Electronics (Info-Comm)}
	{Software Engineer}
	{Defence Electronics Division \\Was introduced and recommended to work on project:
	\begin{itemize}[itemsep=0pt,topsep=0pt]
	\item Search and sort of large data
	\item Categorisation of information
	\item Interfacing with several communications equipment
	\end{itemize}}
\end{entrylist}

\section{education}

\begin{entrylist}
	\entry
	{2009 - 2012}
	{Bachelors of Engineering (BEng)}
	{National University of Singapore}
	{
		Major in Computer Engineering
	}
	\entry
	{2006 - 2009}
	{Diploma in Electronics and Computer Engineering}{Ngee Ann Polytechnic, Singapore}
	{
		Diploma with Merit\\
	}
	\entry
	{2005}
	{GCE 'O' Levels}{Clementi Town Secondary School, Singapore}
	{}
\end{entrylist}
\pagebreak
\section{Notable Projects}
\begin{entrylist}
	\entry{2012}
	{Mobile Ground Station for Unmanned Aerial Vehicles}
	{}
	{
		A Ground Control Station built on the iPad (iOS 6) interfacing for a custom UAV. The UAV took part in the DARPA UAVForge competition. Communication was via Wi-Fi and 3G.

	}
	\entry{2012}
	{Social Care Assistance Network}
	{}
	{Social Network for Android (V15, Ice Cream Sandwich) devices, developed for aiding in quick response to non-emergency situations. Involved the usage of an indoor localization server capable of pinpointing indoor location via power signal of network device.}
	\entry{2012}
	{Networked Bomberman}
	{}
	{A multiplayer clone of the classic Bomberman for the browser and Blackberry Playbook made using CraftyJS (JS game engine) and NodeJS. Drew graphics and implemented game logic. }
	\entry{2010}
	{i8051 Sequencer}
	{}
	{Designed the 8051 sequencer which contains the algorithm for implementing assembly instructions of the Intel 8051 in the Industry Standard Architecture. Used the Spartan 3A Evaluation Kit (FPGA) to test the sequencer logic in VHDL.}
	\entry{2009}
	{eSuite}
	{}
	{A web-based control system in C\# to remotely control and monitor the activities of an indoor environment. Sensor boards, logic circuits, and motor drivers were built from scratch. The suite features audio-video transmission, motion detection, temperature sensors, automatic triggers and alarms via SMS or email.}

\end{entrylist}



\section{Groups}

\begin{entrylist}
	\entry
	{2014}
	{R User Group}
	{Attendee}
	{Avid interest in R Language due to its wide usage by statisticians, for data mining and analysis.}
	
\end{entrylist}
%\printbibsection{article}{article in peer-reviewed journal}
%\begin{refsection}
%	\nocite{*}
%	\printbibliography[sorting=chronological, type=inproceedings, title={international peer-reviewed conferences/proceedings}, notkeyword={france}, heading=subbibliography]
%\end{refsection}
%\begin{refsection}
%	\nocite{*}
%	\printbibliography[sorting=chronological, type=inproceedings, title={local peer-reviewed conferences/proceedings}, keyword={france}, heading=subbibliography]
%\end{refsection}
%\printbibsection{misc}{other publications}
%\printbibsection{report}{research reports}
\section{interests}
uavs, machine learning, information visualisation, arduinos, inline skating, reading, gaming
\end{document}
