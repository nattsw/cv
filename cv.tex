%!TEX TS-program = xelatex
\documentclass[]{friggeri-cv}

\usepackage{marvosym} % Allows the use of symbols
\usepackage{amsmath} % Allows the use of symbols
\usepackage{enumitem}
%\addbibresource{bibliography.bib}

\begin{document}
\header{shawn}{tan}
{aspiring computer scientist}
% In the aside, each new line forces a line break
\begin{aside}
	\section{online}
	\Email~ \href{mailto:shawn@wtf.sg}{shawn@wtf.sg}
	\href{http://blog.wtf.sg}{http://blog.wtf.sg}
	\href{http://facebook.com/shawntan.171287}{fb://shawntan.171287}
	\href{http://github.com/shawntan}{shawntan@github}
	\href{http://sg.linkedin.com/in/tanshawn}{tanshawn@linkedin}
	\section{languages}
	english (proficient)
	mandarin
	\section{programming}
	{\color{red} $\varheartsuit$} JavaScript, Java
	Python, Ruby, C, Perl
	\section{technical skills}
	HTML \& CSS, \LaTeX,
	Linux System Administration
\end{aside}
\section{about}
\begin{tabular}{ p{0.45cm} p{6.5cm} p{0.45cm} p{6.5cm} }
	\Male 	& Tan Jing Shan, Shawn	 & \Sagittarius &  DOB: 17/12/1987\\
	\Letter & 5 Jalan Asas, Singapore 678763 & \Mobilefone & (+65) 93877026 \\
\end{tabular}

\section{experience}

\begin{entrylist}
	\entry
	{2013 - now}
	{Semantics3}
	{Software Engineer}
	{}
	\entry
	{2008 - 2009}
	{Singapore Armed Forces}
	{Manpower Officer}
	{Administrative position managing National Servicemen}
	\entry
	{04 - 06 2007}
	{\href{http://www.np.edu.sg/ict/facilities/rhymes/Pages/loc\_rhymes.aspx}{RHyMeS Center, Ngee Ann Polytechnic}}
	{Research Assistant}
	{Development of a conference management system using active RFID tags for 
	tracking.}
	\entry
	{01 - 03 2007}
	{Ngee Ann Polytechnic}
	{Teaching Assistant}
	{
		\begin{itemize}[itemsep=0pt,topsep=0pt]
			\item Provided technical support for students working on the RHyMeS 
				project.
			\item Facilitated workshops on using the RHyMeS SDK and API.
			\item  Taught students working on their projects how to use the Java 
				Swing UI API
		\end{itemize}
	}
\end{entrylist}

\section{education}

\begin{entrylist}
	\entry
	{2009 - 2012}
	{Bachelors of Computing (BComp)}
	{National University of Singapore}
	{
		Major in Computer Science - Upper 2$^{\text{nd}}$ Class Honours \\
		Special Programme in Computing (Turing Programme) \\
		Focus Area: Artificial Intelligence\\
		CAP: 4.39 / 5
	}
	\entry
	{2004 - 2007}
	{Diploma in Information Technology}{Ngee Ann Polytechnic, Singapore}
	{
		Specialisation in Software Engineering - Diploma with Merit\\
		GPA: 3.8 / 4
	}
	\entry
	{2003}
	{GCE 'O' Levels}{Mayflower Secondary School, Singapore}
	{}
\end{entrylist}
\pagebreak
\section{academic projects}
\begin{entrylist}
	\entry{2010 - 2011}
	{\href{http://wing.comp.nus.edu.sg/portal/publications.html?view=publication&task=show&id=151}{grab\textit{smart}: 
	A User-centric Web Information Extraction System}}
	{}
	{
		An integrated system that allows users to easily select portions of a 
		page to extract, and subsequently extracts the data in a manner that is 
		robust to site layout changes.
	}
	\entry{2012}
	{
		\href
		{http://wing.comp.nus.edu.sg/portal/publications.html?view=publication&task=show&id=178}
		{Predicting Web 2.0 Thread Updates}
	}
	{}
	{
		A method to estimate arrival times of new posts to forum discussion 
		threads and an evaluation metric for measuring the effectiveness of an 
		incremental crawler of a site with time-sensitive data.
	}
	\entry{04 2011}
	{
		\href
		{https://github.com/shawntan/lspi-tetris-agent}
		{LSPI Tetris Agent}
	}
	{}
	{
		Application of the least-squares policy iteration algorithm to play 
		Tetris as part of the CS3243 Introduction to Artificial Intelligence 
		course. The agent can complete 100,000 lines on average, and up to a 
		maximum of about a million lines.
	}
	\entry{04 2012}
	{
		\href
		{https://github.com/shawntan/mm-crawl}
		{Focused Web Crawling using Markov Decision Processes}
	}
	{}
	{
		Using the least-squares policy iteration algorithm and some heuristics	
		to perform focused crawling. For CS4246R AI for Planning and Decision 
		Making (Research Project)
	}
	\entry{10 2011}
	{
		\href
		{https://www.dropbox.com/s/fe9d9io04iv4g8q/cshumour.pdf}
		{Survey Paper: Computational Humour}
	}
	{}
	{
		A literature review of 6 approaches to computational humour. For CS3243R 
		Introduction to Artificial Intelligence (Research Project)
	}

\end{entrylist}



\section{organisations}

\begin{entrylist}
	\entry
	{2010 - 2012}
	{NUS Hackers}
	{Coreteam Member}
	{Duties include facilitating NUS Hacker activities: Hack \& Roll 2012, 
weekly Friday Hacks, and maintaining the download.nus.edu.sg mirror service}
	\entry
	{2003 - 2012}
	{Singapore Scouts Association}{Assistant Scout Leader}
	{Attached to the Mayflower Secondary School Boy Scouts Group. Duties include 
	facilitating camps, weekly meetings, and mentoring both scouts and 
ventures.}
\end{entrylist}
%\printbibsection{article}{article in peer-reviewed journal}
%\begin{refsection}
%	\nocite{*}
%	\printbibliography[sorting=chronological, type=inproceedings, title={international peer-reviewed conferences/proceedings}, notkeyword={france}, heading=subbibliography]
%\end{refsection}
%\begin{refsection}
%	\nocite{*}
%	\printbibliography[sorting=chronological, type=inproceedings, title={local peer-reviewed conferences/proceedings}, keyword={france}, heading=subbibliography]
%\end{refsection}
%\printbibsection{misc}{other publications}
%\printbibsection{report}{research reports}
\section{research interests}
artificial general intelligence, recurrent neural networks, machine learning,\\
reinforcement learning, graphical models, natural language processing, data 
mining
\end{document}
