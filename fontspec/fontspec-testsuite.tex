%%
%% This is file `fontspec-testsuite.tex',
%% generated with the docstrip utility.
%%
%% The original source files were:
%%
%% fontspec.dtx  (with options: `testsuite')
%% 
%%   _________________________________________
%%   The fontspec package for XeLaTeX/LuaLaTeX
%%   (C) 2004--2013    Will Robertson and Khaled Hosny
%% 
%%   License information appended.
%% 
%% 


\documentclass{article}
\usepackage{fontspec,fancyvrb,geometry,graphicx,calc,multicol}
\geometry{margin=2cm,a4paper}
\setmainfont[Ligatures=TeX]{TeX Gyre Pagella}
\setmonofont{Inconsolata}

\newcommand\meta[1]{$\langle$\emph{#1}$\rangle$}

\newcommand\TEST[1]{%
  \section{Test #1}
  \begin{minipage}{0.5\textwidth}
  \VerbatimInput{testsuite/#1.ltx}
  \end{minipage}
  \hfill
  \begin{minipage}{8cm+4pt}
    \fboxsep=1pt
    \fboxrule=1pt
    \fbox{\insertpdf#1\END}
  \end{minipage}
}

\def\insertpdf#1#2\END{%
  \ifnum`#1=70\relax
    \includegraphics[width=8cm]{testsuite/#1#2.Lsafe.pdf}%
  \else
    \includegraphics[width=8cm]{testsuite/#1#2.safe.pdf}%
  \fi
}

\newcommand\codeline[1]{\par{\centering#1\par}\par\noindent\ignorespaces}

\pagestyle{empty}
\begin{document}
\title{The \textsf{fontspec} test suite}
\author{Will Robertson}
\date{Compiled: \today}
\maketitle
\thispagestyle{empty}

\section*{Preamble}

\begin{multicols}{2}\noindent
The examples shown in the remainder of the document are generated directly from the code shown alongside.
As well as being good minimal examples, these tests are useful to ensure that changes to \textsf{fontspec} don't affect old behaviour.

When the test suite is run, the new output is compared pixel by pixel with that shown here and warnings produced if the outputs are not identical (within a small tolerance to account for rounding errors).

Tests with a name that begins with `\textit X' are processed with \texttt{xelatex} only; `\textit L' with \texttt{lualatex}; and `\textit F' with both.
In the latter case, the output from both engines is compared with each other, ensuring that the package is consistent cross-platform.

To generate tests yourself, write a new file in the \texttt{testsuite/} folder with filename
\codeline{\meta{F,L,X}\meta{num}\meta{letter}\texttt{.ltx}}
according to the naming scheme above and the following numbering scheme. If you are writing a new test entirely, increment the \meta{num}; if you are writing a variation on an old test, increment the \meta{letter}. No need to be too fussy, though.

After writing the new test, run
\codeline{\texttt{make initest}}
This will generate the reference output (\texttt{*.safe.pdf}) from which to check all future tests.
After any changes have been made to \textsf{fontspec}, the entire test suite is compiled with \codeline{\texttt{make check}}
If you want to run just a single test, use
\codeline{\texttt{make }\meta{testname}}
(but this syntax \emph{may} be fragile and need to be changed in the future).
\end{multicols}

\input{testsuite/testsuite-listing.tex}

\end{document}
%% 
%% Copyright 2004--2013 Will Robertson <wspr81@gmail.com>
%% Copyright 2009--2013   Khaled Hosny <khaledhosny@eglug.org>
%% 
%% Distributable under the LaTeX Project Public License,
%% version 1.3c or higher (your choice). The latest version of
%% this license is at: http://www.latex-project.org/lppl.txt
%% 
%% This work is "author-maintained" by Will Robertson.
%% 
%% This work consists of this file fontspec.dtx
%%           and the derived files fontspec.sty,
%%                                 fontspec.lua,
%%                                 fontspec.cfg,
%%                                 fontspec-xetex.tex,
%%                                 fontspec-luatex.tex,
%%                             and fontspec.pdf.
%% 
%%
%% End of file `fontspec-testsuite.tex'.
